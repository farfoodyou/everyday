\documentclass{article}
\usepackage{graphicx} %use graph format
\usepackage{epstopdf}
\usepackage{amsmath}
\usepackage{amssymb}

    \author{Hu Yalyu}
    \title{My latex manpage}
\begin{document}
\maketitle
\tableofcontents
\section{Hello China} 
China is in East Asia.\\ 2nd sentence.
% \begin{figure}
%     \centering
%     \includegraphics[height=5cm,width=10cm]{fig.1}
%     \caption{yes}
%     \label{1}
% \end{figure}
    \subsection{Hello Beijing} 
    Beijing is the capital of China.
        \subsubsection
        {Hello Dongcheng District}
\paragraph
{Tian'anmen Square}is in the center of Beijing
\subparagraph
{Chairman Mao} is in the center of Tian'anmen Square
\subsection
{Math Function}
\paragraph
The Newton's second law is F=ma.

The Newton's second law is $F=ma$.

The Newton's second law is
$$F=ma$$

The Newton's second law is
\[F=ma\]

Greek Letters $\eta$ and $\mu$

Fraction $\frac{a}{b}$

Power $a^b$

Subscript $a_b$

Derivate $\frac{\partial y}{\partial t} $

Vector $\vec{n}$

Bold $\mathbf{n}$

To time differential $\dot{F}$

Matrix (lcr here means left, center or right for each column)
\[
\left[
\begin{array}{lcr}
a1 & b22 & c333 \\
d444 & e555555 & f6
\end{array}
\right]
\]

Equations(here \& is the symbol for aligning different rows)
\begin{align}
a+b&=c\\
d&=e+f+g
\end{align}

\[
\left\{
\begin{aligned}
&a+b=c\\
&d=e+f+g
\end{aligned}
\right.
\]

\end{document}
